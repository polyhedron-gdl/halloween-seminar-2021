\documentclass[11pt]{beamer}
\usetheme{CambridgeUS}
\usepackage[utf8]{inputenc}
\usepackage{amsmath}
\usepackage{amsfonts}
\usepackage{amssymb}
\usepackage{graphicx}
\usepackage{pgfpages}
\usepackage{framed}
\usepackage{xcolor}
\usepackage[most]{tcolorbox}
\usepackage{soul}
\usepackage{empheq}

% The replacement character � (often displayed as a black rhombus with a white
% question mark) is a symbol found in the Unicode standard at code point U
% +FFFD in the Specials table. It is used to indicate problems when a system 
% is unable to render a stream of data to a correct symbol.[4] It is usually 
% seen when the data is invalid and does not match any character. For this 
% reason we map explicitly this character to a blanck space.
\DeclareUnicodeCharacter{FFFD}{ }

\newcommand*{\itemimg}[1]{%
  \raisebox{-.3\baselineskip}{%
    \includegraphics[
      height=\baselineskip,
      width=\baselineskip,
      keepaspectratio,
    ]{#1}%
  }%
}

\newtcbox{\mymath}[1][]{%
    nobeforeafter, math upper, tcbox raise base,
    enhanced, colframe=blue!30!black,
    colback=blue!10, boxrule=1pt,
    #1}

\newcommand{\highlight}[1]{%
  \colorbox{yellow!100}{$\displaystyle#1$}}

\author{Giovanni Della Lunga\\{\footnotesize giovanni.dellalunga@unibo.it}}
%\title{3 - Introduction to Deep Learning}
%\title{4 - Basic Text Analysis}
%\title{5 - Introduction to Natural Language Processing}
%\title{7 - Classification for Text Analysis}
\title{8 - Clustering for Text Similarity}
%\title{9 - Information Extraction}
\subtitle{} % (optional)
\setbeamercovered{transparent} 
\institute{Halloween Conference in Quantitative Finance} 
\date{Bologna - October 26-28, 2021} 

\begin{document}

\begin{frame}
\includegraphics[width=\linewidth]{img/halloween-seminar-logo.PNG}
\end{frame}

\begin{frame}
\titlepage
\end{frame}

\AtBeginSection[]
{
  %\begin{frame}<beamer>
  %\footnotesize	
  %\frametitle{Outline}
  %\begin{multicols}{2}
  %\tableofcontents[currentsection]
  %\end{multicols}	  
  %\normalsize
  %\end{frame}
  \begin{frame}
  \vfill
  \centering
  \begin{beamercolorbox}[sep=8pt,center,shadow=true,rounded=true]{title}  	\usebeamerfont{title}\insertsectionhead\par%
  \end{beamercolorbox}
  \vfill
  \end{frame}
}
\AtBeginSubsection{\frame{\subsectionpage}}

% INSERT HERE
\subsection{Similarity \\ \scalebox{0.8}{Sorting documents}}
%---------------------------------------------------------------------------------------------------
\begin{frame}{Similarity}
	\begin{itemize}
		\item At its core, any sorting task relies on our ability to compare two documents and determine their \textbf{SIMILARITY}. 
	\end{itemize}
	\begin{center}
	\includegraphics[scale=0.5]{../../07-pictures/08_clustering_for_text_similarity_pic_0.png}
	\end{center}
\end{frame}
%..................................................................
\begin{frame}{Introduction}
\begin{columns}[T] % align columns
\begin{column}{.48\textwidth}
        \begin{itemize}
		\item Documents that are similar to each other are grouped together and the resulting groups broadly describe the overall themes, topics, and patterns inside the corpus.
        \end{itemize}
\end{column}%
\hfill%
\begin{column}{.48\textwidth}
    %\fbox{
        \includegraphics[width=\linewidth]{../../07-pictures/08_clustering_for_text_similarity_pic_1.png}
    %}
\end{column}%
\end{columns}
\end{frame}
\begin{frame}{Introduction}
\begin{columns}[T] % align columns
\begin{column}{.48\textwidth}
        \begin{itemize}
		\item While most document sorting is currently done manually, it is possible to achieve these tasks in a fraction of the time with the effective integration of unsupervised learning, as we will see in this lesson
        \end{itemize}
\end{column}%
\hfill%
\begin{column}{.48\textwidth}
    %\fbox{
        \includegraphics[width=\linewidth]{../../07-pictures/08_clustering_for_text_similarity_pic_2.png}
    %}
\end{column}%
\end{columns}
\end{frame}
\begin{frame}{Introduction}
	\begin{itemize}
		\item In many situations, corpora do not arrive \textit{pretagged} with labels ready for classification. 
		\item In these cases, the only choice, or at least a necessary precursor for many natural language processing tasks, is an unsupervised approach. 
		\item Clustering algorithms aim to discover \textbf{latent structure} or themes in unlabeled data using features to organize instances into meaningfully dissimilar groups.
		\item With text data, each \textbf{ instance} is a single document or sentence, and the \textbf{features} are its tokens, vocabulary, structure, metadata, etc.
	\end{itemize}
\end{frame}
%..................................................................
\begin{frame}{Unsupervised Learning on Text}
	\begin{itemize}
		\item \textbf{Comparison between Clustering and Classification}
		\item The behavior of unsupervised learning methods is fundamentally different from that of supervised algorithm we have seen in the previous section; 
		\item \textbf{Instead of learning a predefined pattern, the model attempts to find relevant patterns a priori}.
		\item A clustering algorithm is usually employed to create groups or topic clusters, using a \underline{distance metric} such that documents that are closer together in feature space are more similar. 
		\item New incoming documents can then be vectorized and assigned to the nearest cluster.
	\end{itemize}
\end{frame}
%..................................................................
\begin{frame}{Unsupervised Learning on Text}
\begin{columns}[T] % align columns
\begin{column}{.48\textwidth}
        \begin{itemize}
		\item A corpus is transformed into feature vectors and a clustering algorithm is employed to create groups or topic clusters, using a distance metric such that documents that are closer together in feature space are more similar. 
		\item New incoming documents can then be vectorized and assigned to the nearest cluster.
        \end{itemize}
\end{column}%
\hfill%
\begin{column}{.48\textwidth}
    %\fbox{
        \includegraphics[width=\linewidth]{../../07-pictures/08_clustering_for_text_similarity_pic_3.png}
    %}
\end{column}%
\end{columns}
\end{frame}
\begin{frame}{Unsupervised Learning on Text}
	\begin{itemize}
		\item As we have seen in section 3.1, there are a number of different measures that can be used to determine document similarity; 
		\item Remember that, fundamentally, each relies on our ability to imagine documents as points in space, where the relative closeness of any two documents is a measure of their similarity.
		\item We have discussed the \textbf{cosine similarity} but there are others measure of distance we can use in clustering documents. 
	\end{itemize}
\end{frame}
%..................................................................
\begin{frame}{Unsupervised Learning on Text : Similarity}
	\begin{center}
	\includegraphics[scale=0.4]{../../07-pictures/08_clustering_for_text_similarity_pic_4.png}
	\end{center}
\end{frame}
%..................................................................
\begin{frame}{Document Similarity}
\begin{columns}[T] % align columns
\begin{column}{.48\textwidth}
        \begin{itemize}
		\item We can measure vector similarity with cosine distance, using the cosine of the angle between the two vectors to assess the degree to which they share the same orientation. 
		\item In effect, the more parallel any two vectors are, the more similar the documents will be (regardless of their magnitude).
        \end{itemize}
\end{column}%
\hfill%
\begin{column}{.48\textwidth}
    %\fbox{
        \includegraphics[width=\linewidth]{../../07-pictures/08_clustering_for_text_similarity_pic_5.png}
    %}
\end{column}%
\end{columns}
\end{frame}
\subsection{Clustering \\ \scalebox{0.8}{}}
%---------------------------------------------------------------------------------------------------
\begin{frame}{Clustering}
	\begin{itemize}
		\item Now that we can quantify the similarity between any two documents, we can begin exploring unsupervised methods for finding similar groups of documents. 
		\item \textbf{Partitive clustering} and \textbf{agglomerative clustering} are our two main approaches, and both separate documents into groups whose members share maximum similarity as defined by some distance metric.
		\item \textbf{We will focus on partitive methods, which partition instances into groups that are represented by a central vector (the centroid)} or described by a density of documents per cluster. 
		\item Centroids represent an aggregated value (e.g., mean or median) of all member documents and are a convenient way to describe documents in that cluster.
	\end{itemize}
\end{frame}
%..................................................................
\begin{frame}{Clustering}
	\begin{itemize}
		\item As we have seen in the first part, clustering can be considered one of the most important unsupervised learning problem; so, as every other problem of this kind, it deals with finding a structure in a collection of unlabeled data. 
		\item A loose definition of clustering could be \textit{the process of organizing objects into groups whose members are similar in some way}. 
		\item A cluster is therefore a collection of objects which are coherent internally, but clearly dissimilar to the objects belonging to other clusters.
	\end{itemize}
\end{frame}
%..................................................................
\begin{frame}{Text Documents Clustering using K-Means Algorithm }
	\begin{center}
	\includegraphics[scale=0.5]{../../07-pictures/08_clustering_for_text_similarity_pic_6.png}
	\end{center}
	\footnotesize{source: \textit{https://www.codeproject.com/Articles/439890/Text-Documents-Clustering-using-K-Means-Algorithm}}
\end{frame}
%..................................................................
\begin{frame}{Text Documents Clustering using K-Means Algorithm}
	\begin{itemize}
		\item \textbf{Document Representation}
		\item \highlight{\text{Each document is represented as a vector}} using the vector space model. 
		\item The vector space model also called term vector model is an algebraic model for representing text document (or any object, in general) as vectors of identifiers. \highlight{\text{For example, TF-IDF weight}}.
		\item \textbf{Finding Similarity Score}
		\item We will use cosine similarity to identify the similarity score of a document. 
	\end{itemize}
\end{frame}
%..................................................................
\begin{frame}{Text Documents Clustering using K-Means Algorithm}
	\begin{itemize}
		\item \textbf{A Practical Example: Clustering Movie Reviews}
		\item We are going to use data collected by Brandon Rose; 
		\item here the link to the original post: http://brandonrose.org/top100
		\item And the GitHub Repository in which you can find all the data files necessary for this example: https://github.com/brandomr/document\textunderscore cluster
	\end{itemize}
\end{frame}
%..................................................................
\begin{frame}{Text Documents Clustering using K-Means Algorithm}
	\textbf{5 Steps}
	\begin{itemize}
		\item Prepare data
		\item Tokenizing and stemming each synopsis
		\item Transforming the corpus into vector space using tf-idf
		\item Calculating cosine distance between each document as a measure of similarity
		\item Clustering the documents using the k-means algorithm
	\end{itemize}
\end{frame}
%..................................................................
\begin{frame}{Clustering}
\begin{columns}[T] % align columns
\begin{column}{.48\textwidth}
        \begin{itemize}
		\item Working Example
		\item Using \textbf{08-clustering-for-text-similarity.ipynb} Notebook 
		\item Clustering Film Reviews
		\item Clustering Wikipedia
        \end{itemize}
\end{column}%
\hfill%
\begin{column}{.48\textwidth}
    %\fbox{
        \includegraphics[width=\linewidth]{../../07-pictures/08_clustering_for_text_similarity_pic_7.png}
    %}
\end{column}%
\end{columns}
\end{frame}
\begin{frame}{Clustering}
	\begin{itemize}
		\item In the previous section, we explored partitive methods, which divide points into clusters. 
		\item By contrast, hierarchical clustering involves creating clusters that have a predetermined ordering from top to bottom. 
		\item Hierarchical models can be either \textbf{agglomerative}, where clusters begin as single instances that iteratively aggregate by similarity until all belong to a single group, or \textbf{divisive}, where the data are gradually divided, beginning with all instances and finishing as single instances.
	\end{itemize}
\end{frame}
%..................................................................
\begin{frame}{Hierarchical clustering}
	\begin{center}
	\includegraphics[scale=0.5]{../../07-pictures/08_clustering_for_text_similarity_pic_8.png}
	\end{center}
\end{frame}
%..................................................................
\begin{frame}{Hierarchical clustering}
	\begin{itemize}
		\item Agglomerative clustering iteratively combines the closest instances into clusters until all the instances belong to a single group. 
		\item In the context of text data, the result is a hierarchy of variable-sized groups that describe document similarities at different levels or granularities.
		\item Just as there are multiple ways of quantifying the difference between any two documents, there are also multiple criteria for establishing the linkages between them. 
		\item Agglomerative clustering requires both a \textbf{distance function} and a \textbf{linkage criterion}. 
		\item Scikit-Learn implementation defaults to the Ward criterion, which minimizes the within-cluster variance as each are successively merged. 
		\item At each aggregation step, the algorithm finds the pair of clusters that contributes the least increase in total within-cluster variance after merging.
	\end{itemize}
\end{frame}
%..................................................................
%=====================================================================


\end{document}
