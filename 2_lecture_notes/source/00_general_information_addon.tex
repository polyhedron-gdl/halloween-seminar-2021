\documentclass[11pt]{beamer}
\usetheme{CambridgeUS}
\usepackage[utf8]{inputenc}
\usepackage{amsmath}
\usepackage{amsfonts}
\usepackage{amssymb}
\usepackage{graphicx}
\usepackage{pgfpages}
\usepackage{framed}
\usepackage{xcolor}
\usepackage[most]{tcolorbox}
\usepackage{soul}
\usepackage{empheq}

% The replacement character � (often displayed as a black rhombus with a white
% question mark) is a symbol found in the Unicode standard at code point U
% +FFFD in the Specials table. It is used to indicate problems when a system 
% is unable to render a stream of data to a correct symbol.[4] It is usually 
% seen when the data is invalid and does not match any character. For this 
% reason we map explicitly this character to a blanck space.
\DeclareUnicodeCharacter{FFFD}{ }

\newcommand*{\itemimg}[1]{%
  \raisebox{-.3\baselineskip}{%
    \includegraphics[
      height=\baselineskip,
      width=\baselineskip,
      keepaspectratio,
    ]{#1}%
  }%
}

\newtcbox{\mymath}[1][]{%
    nobeforeafter, math upper, tcbox raise base,
    enhanced, colframe=blue!30!black,
    colback=blue!10, boxrule=1pt,
    #1}

\newcommand{\highlight}[1]{%
  \colorbox{yellow!50}{$\displaystyle#1$}}

\author{Giovanni Della Lunga}
\title{Introduction to Natural Language Processing}
\subtitle{} % (optional)
\setbeamercovered{transparent} 
%\setbeamertemplate{navigation symbols}{} 
%\logo{} 
%\institute{} 
%\date{} 
%\subject{} 
\begin{document}

\begin{frame}
\titlepage
\end{frame}

\AtBeginSection[]
{
  %\begin{frame}<beamer>
  %\footnotesize	
  %\frametitle{Outline}
  %\begin{multicols}{2}
  %\tableofcontents[currentsection]
  %\end{multicols}	  
  %\normalsize
  %\end{frame}
  \begin{frame}
  \vfill
  \centering
  \begin{beamercolorbox}[sep=8pt,center,shadow=true,rounded=true]{title}  	\usebeamerfont{title}\insertsectionhead\par%
  \end{beamercolorbox}
  \vfill
  \end{frame}
}
\AtBeginSubsection{\frame{\subsectionpage}}

% INSERT HERE
%===================================================================================================
\section{Unstructured Data \\ \scalebox{0.8}{}}
%---------------------------------------------------------------------------------------------------
\subsection{Why? \\ \scalebox{0.8}{The purpose and the rationale for the subject covered in the seminar}}
%---------------------------------------------------------------------------------------------------
\begin{frame}{Why?}
	\begin{itemize}
		\item When we think of financial data, our thinking almost always ends up displaying infinite historical series of numbers (prices, interest rates, volatility) 
		\item Financial data usually come in as structured data. 
	\end{itemize}
	\begin{center}
	\includegraphics[scale=1]{../../07-pictures/00_general_information_addon_pic_0.png}
	\end{center}
\end{frame}
%..................................................................
\begin{frame}{The importance of Unstructured Data}
	 On the other hand, unstructured data, such as call transcripts, emails text, transactional data are largely an area not yet fully exploited due to accessibility and processing challenges.
	\begin{center}
	\includegraphics[scale=1]{../../07-pictures/00_general_information_addon_pic_1.png}
	\end{center}
\end{frame}
%..................................................................
\begin{frame}{The importance of Unstructured Data}
	\begin{center}
	\includegraphics[scale=1]{../../07-pictures/00_general_information_addon_pic_2.png}
	\end{center}
\end{frame}
%..................................................................
\subsection{What? \\ \scalebox{0.8}{The key content, principles and topics to be learned in these lessons}}
%---------------------------------------------------------------------------------------------------
\subsection{How? \\ \scalebox{0.8}{the learning tools we are going to use}}
%---------------------------------------------------------------------------------------------------
\begin{frame}{How?}
	\begin{center}
	\includegraphics[scale=1]{../../07-pictures/00_general_information_addon_pic_3.png}
	\end{center}
\end{frame}
%..................................................................
%=====================================================================


\end{document}
